a)
Consulting an Erlang table, $m \leq 8$, hence $m_{min} = 8$ channels. The mean
idle time is $\overline{T}_{idle} = \frac{1}{\lambda} = 1$ minute.

b)
If we assume that the calls are directed randomly to each cell, the effective
offered load for each cell is $a_1 = a_2 = 2.5$ Erlangs, which means that
$m_{min,1} = m_{min,2} = m_{min} = 5$ channels. The resulting mean idle time for
the two-cell configuration is
$\overline{T}_1 = \overline{T}_2 = \frac{1}{0.5} = 2$ minutes.

c)
Assuming that we need the minimum amount of channels for the fulfilment of the
blocking probability requirement, 8 channels are needed in the first case, and
10 in the second one. Hence, it is natural to assume that the second option is
more expensive to build, although it would be more energy-efficient due to its
lengthier mean idle time.
