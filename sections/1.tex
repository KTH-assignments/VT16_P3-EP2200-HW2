The first packet will wait, on average, the sum of the interarrival times between
it and the $50^{th}$ packet. Since the packets arrive in a Poisson fashion, the
interarrival times are exponentially distributed. Due to the memoryless property
of the exponential distribution, the mean interarrival time between packets is
$$E[T] = \dfrac{1}{\lambda} = 1\ \text{ms}$$

Hence, the mean waiting time for the first packet is
$$\overline{T}_w = \sum_{i=1}^{50-1} E[T] = \dfrac{49}{\lambda} = 49\ \text{ms}$$

The probability that $k$ packets are transmitted in a time interval of
$\Delta t = 10\ \text{ms}$ is
$$P_k = \dfrac{(\lambda \Delta t)^k}{k!}e^{-\lambda \Delta t} = \dfrac{10^k}{k!}e^{-10}$$

The average number of packets arriving inside an interval of
$\Delta t = 10\ \text{ms}$ is
$$E[P] = \lambda \Delta t = 10\ \text{packets / block}$$

Hence the average number of packets in a block is $\overline{B} = 1 + E[P] = 11$ packets / block,
since the system waits for 10 ms after the arrival of the first packet of the block.
