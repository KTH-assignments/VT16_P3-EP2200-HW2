For the missing pieces of the diagonal of the state transition intensity matrix,
the following equation holds:
$$q_{i,i} = -\sum_{i \neq j} q_{i,j}$$

Hence matrix $Q$ is formed as

\begin{equation*}
  Q =
  \begin{bmatrix}
    -2  & 1   & 0   & 1   \\
    1   & -4  & 3   & 0   \\
    0   & 3   & -3  & 0   \\
    0   & 3   & 0   & -3  \\
  \end{bmatrix}
\end{equation*}

\begin{figure}[H]
	\centering
	\scalebox{1}{\begin{tikzpicture}[->,>=stealth',shorten >=1pt,auto,node distance=3cm,
                    thick,main node/.style={circle,draw}]

  \node[main node] (1) {1};
  \node[main node] (2) [right of=1] {2};
  \node[main node] (3) [below of=2] {3};
  \node[main node] (4) [below of=1] {4};

  \path[every node]
    (1) edge [bend left] node[above] {1} (2)
        edge [bend right] node[left] {1} (4)

    (2) edge [bend left] node[above] {1} (1)
        edge [bend right] node[left] {3} (3)

    (3) edge [bend right] node[right] {3} (2)

    (4) edge node [left] {3} (2);
\end{tikzpicture}
}
	\caption{The Markovian chain corresponding to the above $Q$ matrix}
\end{figure}

In steady state $[\pi_1\ \pi_2\ \pi_3\ \pi_4] \cdot \mathbf{Q} = \mathbf{0}$,
where $\pi_i$ is the probability of the radio being in state $i$. Solving the
system of equations gives
$$[\pi_1\ \pi_2\ \pi_3\ \pi_4] = [\frac{3}{16}\ \frac{3}{8}\ \frac{3}{8}\ \frac{1}{16}]$$

Hence there is a probability of $\frac{3}{16}=18.75\%$ that the radio is in
stand-by, $\frac{3}{8}=37.5\%$ that the radio is listening to the radio channel
for incoming packets, or receiving a packet, and $\frac{1}{16} = 6.25\%$ that
the radio is transmitting a packet.

This means that if we observe the state distribution of the radio over a period
of time $T$, then as $T$ approaches infinity, we will observe that the radio is
in state 1 for $0.1875T$ time units, in state 2 for $0.375T$ time units, as is
also the case for the time spent in state 3, and in state 4 for $0.0625T$ time
units.
