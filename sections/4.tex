a)
\begin{figure}[H]
	\centering
	\scalebox{0.8}{% Generated with LaTeXDraw 2.0.8
% Mon Feb 08 02:42:40 CET 2016
% \usepackage[usenames,dvipsnames]{pstricks}
% \usepackage{epsfig}
% \usepackage{pst-grad} % For gradients
% \usepackage{pst-plot} % For axes
\scalebox{1} % Change this value to rescale the drawing.
{
\begin{pspicture}(0,-0.83)(12.88,0.83)
\psframe[linewidth=0.04,dimen=outer](8.82,0.67)(2.36,-0.63)
\pscircle[linewidth=0.04,dimen=outer](9.63,0.0){0.83}
\psline[linewidth=0.04cm,arrowsize=0.05291667cm 2.0,arrowlength=1.4,arrowinset=0.4]{->}(10.46,0.01)(12.86,0.01)
\psline[linewidth=0.04cm,arrowsize=0.05291667cm 2.0,arrowlength=1.4,arrowinset=0.4]{->}(0.0,0.03)(2.4,0.03)
\psline[linewidth=0.04cm](8.32,0.63)(8.32,-0.61)
\psline[linewidth=0.04cm](8.1,0.63)(8.1,-0.61)
\psline[linewidth=0.04cm](8.54,0.63)(8.54,-0.61)
\psline[linewidth=0.04cm](7.7,0.63)(7.7,-0.61)
\psline[linewidth=0.04cm](7.48,0.63)(7.48,-0.61)
\psline[linewidth=0.04cm](7.92,0.63)(7.92,-0.61)
\psline[linewidth=0.04cm](7.1,0.63)(7.1,-0.61)
\psline[linewidth=0.04cm](6.88,0.63)(6.88,-0.61)
\psline[linewidth=0.04cm](7.32,0.63)(7.32,-0.61)
\psline[linewidth=0.04cm](6.48,0.63)(6.48,-0.61)
\psline[linewidth=0.04cm](6.26,0.63)(6.26,-0.61)
\psline[linewidth=0.04cm](6.7,0.63)(6.7,-0.61)
\psdots[dotsize=0.12](5.96,0.01)
\psdots[dotsize=0.12](5.52,0.01)
\psdots[dotsize=0.12](5.1,0.01)
\end{pspicture}
}

}
	\caption{The block diagram of the queue. There are infinite places in the
    queue and only one server.}
\end{figure}

\begin{figure}[H]
	\centering
	\scalebox{1}{\begin{tikzpicture}[->,>=stealth',shorten >=1pt,auto,node distance=3cm,
                    thick,main node/.style={circle,draw}]

  \node[main node] (0) {0};
  \node[main node] (1) [right of=0] {1};
  \node[main node] (2) [right of=1] {2};
  \node[main node] (3) [right of=2] {3};
  \node[draw=none] (ellipsis1) at (10,0) {$\cdots$};

  \path[every node]
    (0) edge [bend left] node[above] {$\lambda$}  (1)

    (1) edge [bend left] node[below] {$\mu$}      (0)
        edge [bend left] node[above] {$\lambda$}  (2)

    (2) edge [bend left] node[below] {$\mu$}      (1)
        edge [bend left] node[above] {$\lambda$}  (3)

    (3) edge [bend left] node[below] {$\mu$}      (2);

\end{tikzpicture}
}
	\caption{The Markovian chain describing the system}
\end{figure}

b)
A state is defined by the number of packets in the system. Considering local
balance equations we get

\begin{align*}
  \lambda \pi_0 &= \mu \pi_1 \Leftrightarrow \pi_1 = \dfrac{\lambda}{\mu}\pi_0\\
  \lambda \pi_1 &= \mu \pi_2 \Leftrightarrow \pi_2 = \dfrac{\lambda}{\mu}\pi_1 = \Big(\dfrac{\lambda}{\mu}\Big)^2\pi_0\\
  \lambda \pi_2 &= \mu \pi_3 \Leftrightarrow \pi_3 = \Big(\dfrac{\lambda}{\mu}\Big)^3\pi_0\\
  \cdots
\end{align*}

In other words, in general $$\pi_i = \Big(\dfrac{\lambda}{\mu}\Big)^i\pi_0$$

In order to identify all $\pi_i$ we resort to the general law of
$$\sum_{i=0}^{\infty} \pi_i = 1$$

from where we calculate that
$$\pi_0 = 1 - \dfrac{\lambda}{\mu}$$

and

$$\pi_i = \Big(\dfrac{\lambda}{\mu}\Big)^i(1 - \dfrac{\lambda}{\mu})$$

The values for the intensities $\lambda$ and $\mu$ can be derived from

\begin{align*}
  \lambda &= \dfrac{1}{2 \cdot 10^{-3}} = 500 \\
  \mu     &= \dfrac{1}{10^{-3}} = 1000
\end{align*}

Hence the probability that the system is empty is given by

$$\pi_0 = 1 - \dfrac{\lambda}{\mu} = 0.5$$

and, in general, the probability that there are $i$ packets in the system
(in the queue and in the server) is

$$\pi_i = \Big(\dfrac{\lambda}{\mu}\Big)^i(1 - \dfrac{\lambda}{\mu}) = 0.5^{i+1}$$


The average number of packets waiting for transmission $N_q$ is the average
number of packets in the queue, which is

\begin{equation}
  N_q = N - N_s = \sum_{i=1}^{\infty} (i-1) p_i
  \label{eq:4.1}
\end{equation}

where $N$ is the average number of packets in the system, and $N_s$ is the
average number of packets being transmitted. Hence, finding $N_q$ means finding
$N$ and $N_s$.

The average number of packets in the system is defined as

\begin{align*}
  N &= \sum_{i=0}^{\infty} i \cdot \pi_i = \sum_{i=0}^{\infty} i \Big(\dfrac{\lambda}{\mu}\Big)^i \pi_0 =
  \pi_0 \sum_{i=0}^{\infty} i \Big(\dfrac{\lambda}{\mu}\Big)^i \\
  ~ &= \pi_0 \Big(\dfrac{\lambda}{\mu}\Big) \sum_{i=0}^{\infty} i \Big(\dfrac{\lambda}{\mu}\Big)^{i-1} =
  \pi_0 \rho \sum_{i=0}^{\infty} i \rho^{i-1} \\
  ~ &= \pi_0 \rho \sum_{i=0}^{\infty} \dfrac{d(\rho^i)}{d\rho} = \pi_0 \rho \dfrac{d}{d\rho} \Big(\sum_{i=0}^{\infty} \rho^i \Big) \\
  ~ &= \pi_0 \rho \dfrac{d}{d\rho} \Big(\dfrac{1}{1-\rho}\Big) = (1-\rho) \rho \dfrac{1}{(1-\rho)^2} = \dfrac{\rho}{1-\rho} \\
  ~ &= \dfrac{\lambda}{\mu - \lambda} = \dfrac{500}{1000-500} = 1\ \text{packet}
\end{align*}

The average number of packets under transmission is
$$N_s = \dfrac{\lambda}{\mu} = \dfrac{500}{1000} = 0.5\ \text{packets}$$

Hence, from equation \ref{eq:4.1}:

$$N_q = 1 - 0.5 = 0.5\ \text{packets}$$

c)
The probability of at least $n$ packets existing in the system is equal to that
of 1 minus the probability of at most $n-1$ packets existing in the system:
$$P(k \geq n) = 1 - P(k < n) = 1 - P(k \leq n-1)$$

But
\begin{align*}
  P(k \leq n-1) &= \sum_{k=0}^{n-1} \pi_i = \sum_{k=0}^{n-1} (1-\dfrac{\lambda}{\mu})\Big(\dfrac{\lambda}{\mu}\Big)^k \\
  ~             &= (1-\dfrac{\lambda}{\mu})\sum_{k=0}^{n-1} \Big(\dfrac{\lambda}{\mu}\Big)^k =
                   (1-\dfrac{\lambda}{\mu}) \dfrac{1-\Big(\dfrac{\lambda}{\mu}\Big)^n}{1-\dfrac{\lambda}{\mu}} \\
  ~             &= 1-\Big(\dfrac{\lambda}{\mu}\Big)^n
\end{align*}

Hence
$$P(k \geq n) = 1 - P(k \leq n-1) = \Big(\dfrac{\lambda}{\mu}\Big)^n = 0.5^n$$

Intuitively, this makes sense, since the probability of the system having at
least 0 packets is 1, and as the number of packets in the condition increases,
the probability decreases, since the system is stable.

d)
The arriving packet will have to wait for transmission for the duration of $k$
transmission times, supposing that $k$ packets are in the system, $1$ being
transmitted and $k-1$ waiting for transmission in the queue. These times are
distributed exponentially. Assuming $X_i, 1 \leq i \leq k$ is a random variable,
independent of all $X_j, i \neq j$ describing the service time of the $i^{th}$
packet, then the waiting time for packet $k+1$ will be the sum of $X_i$:
$$W = X_1 + X_2 + \dots + X_k = \sum_{i=1}^{k}X_i$$

The addition of $X_1$, although the first packet is assumed to be already
under transmission, is made because of the memoryless property of the exponential
distribution.

This summation of random variables denotes their convolution in time, or
multiplication in the (continuous) Laplace domain. The probability density
function of $X_i$ is
$$f_{X_i}(t) = \mu e^{-\mu t}$$

and its Laplace transform is

$$f^{*}_{T_i}(s) = \mu \dfrac{1}{s+\mu}$$

Hence, the probability density function of the waiting time $W$ of packet $k+1$ in
the Laplace domain, given that there are $k$ packets in the system is

$$f_W(s | k) = \prod_{i=1}^k f^{*}_{T_i}(s) = \prod_{i=1}^k \dfrac{\mu}{s+\mu} = \Big(\dfrac{\mu}{s+\mu}\Big)^k$$

The probability density function of the waiting time $W$ will thus be the
marginalization over all possible values of $k$, starting from $k=1$ because
in order for waiting time to exist, there should be one packet under
transmission in the server

\begin{align*}
  f_W(s) &= \sum_{k=1}^{\infty} \Big(\dfrac{\mu}{s+\mu}\Big)^k \alpha_k = \sum_{k=1}^{\infty} \Big(\dfrac{\mu}{s+\mu}\Big)^k \pi_k \\
         &= \sum_{k=1}^{\infty} \Big(\dfrac{\mu}{s+\mu}\Big)^k (1-\rho) \rho^k = (1-\rho) \sum_{k=1}^{\infty} \Big(\dfrac{\lambda}{s+\mu}\Big)^k \\
         &= (1-\rho) \Big(-1 + \sum_{k=0}^{\infty} \Big(\dfrac{\lambda}{s+\mu}\Big)^k\Big) = (1-\rho) \Big(-1 + \dfrac{1}{1-\dfrac{\lambda}{s+\mu}}\Big) \\
         &= (1-\rho)\Big(-1 + \dfrac{s+\mu}{s+\mu-\lambda}\Big) = (1-\rho) \dfrac{\lambda}{s+\mu-\lambda}
\end{align*}

where the arriving packet $k+1$ observes the system being in state $k$ with
probability $\alpha_k$ is equal to the probability that the system is indeed in
state $k$, $\pi_k$, due to the PASTA property.

Turning $f_W(s)$ in the time domain, we get

\begin{align*}
  f_W(t) &= (1-\rho) \lambda e^{-(\mu - \lambda)t} = (1-\dfrac{\lambda}{\mu}) \lambda e^{-(\mu - \lambda)t} \\
         &= \rho(\mu - \lambda) e^{-(\mu - \lambda)t}
\end{align*}

The CDF of the waiting time $W$ for packet $k+1$ is thus

$$F_W(t) = P(W \leq t) = 1 - \rho e^{-(\mu - \lambda)t}$$

since $F_W(\infty) = 1$ and $F_W(0) = 1-\rho = \pi_0$

The probability that an arriving packet waits in the queue more than $T = 5$ms
before being transmitted is

\begin{align*}
  P(W > T) &= 1 - P(W \leq T) = 1- (1-\rho e^{-(\mu - \lambda)T}) = \dfrac{\lambda}{\mu}e^{-(\mu - \lambda)T} \\
           &= \dfrac{500}{1000} e^{-(1000 - 500) 5 \cdot 10^{-3}} = 0.5 e^{-2.5} = 0.041
\end{align*}
