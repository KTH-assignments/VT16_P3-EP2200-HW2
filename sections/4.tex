\begin{figure}[H]
	\centering
	\scalebox{0.8}{% Generated with LaTeXDraw 2.0.8
% Mon Feb 08 02:42:40 CET 2016
% \usepackage[usenames,dvipsnames]{pstricks}
% \usepackage{epsfig}
% \usepackage{pst-grad} % For gradients
% \usepackage{pst-plot} % For axes
\scalebox{1} % Change this value to rescale the drawing.
{
\begin{pspicture}(0,-0.83)(12.88,0.83)
\psframe[linewidth=0.04,dimen=outer](8.82,0.67)(2.36,-0.63)
\pscircle[linewidth=0.04,dimen=outer](9.63,0.0){0.83}
\psline[linewidth=0.04cm,arrowsize=0.05291667cm 2.0,arrowlength=1.4,arrowinset=0.4]{->}(10.46,0.01)(12.86,0.01)
\psline[linewidth=0.04cm,arrowsize=0.05291667cm 2.0,arrowlength=1.4,arrowinset=0.4]{->}(0.0,0.03)(2.4,0.03)
\psline[linewidth=0.04cm](8.32,0.63)(8.32,-0.61)
\psline[linewidth=0.04cm](8.1,0.63)(8.1,-0.61)
\psline[linewidth=0.04cm](8.54,0.63)(8.54,-0.61)
\psline[linewidth=0.04cm](7.7,0.63)(7.7,-0.61)
\psline[linewidth=0.04cm](7.48,0.63)(7.48,-0.61)
\psline[linewidth=0.04cm](7.92,0.63)(7.92,-0.61)
\psline[linewidth=0.04cm](7.1,0.63)(7.1,-0.61)
\psline[linewidth=0.04cm](6.88,0.63)(6.88,-0.61)
\psline[linewidth=0.04cm](7.32,0.63)(7.32,-0.61)
\psline[linewidth=0.04cm](6.48,0.63)(6.48,-0.61)
\psline[linewidth=0.04cm](6.26,0.63)(6.26,-0.61)
\psline[linewidth=0.04cm](6.7,0.63)(6.7,-0.61)
\psdots[dotsize=0.12](5.96,0.01)
\psdots[dotsize=0.12](5.52,0.01)
\psdots[dotsize=0.12](5.1,0.01)
\end{pspicture}
}

}
	\caption{The block diagram of the queue. There are infinite places in the
    queue and only one server.}
\end{figure}

\begin{figure}[H]
	\centering
	\scalebox{1}{\begin{tikzpicture}[->,>=stealth',shorten >=1pt,auto,node distance=3cm,
                    thick,main node/.style={circle,draw}]

  \node[main node] (0) {0};
  \node[main node] (1) [right of=0] {1};
  \node[main node] (2) [right of=1] {2};
  \node[main node] (3) [right of=2] {3};
  \node[draw=none] (ellipsis1) at (10,0) {$\cdots$};

  \path[every node]
    (0) edge [bend left] node[above] {$\lambda$}  (1)

    (1) edge [bend left] node[below] {$\mu$}      (0)
        edge [bend left] node[above] {$\lambda$}  (2)

    (2) edge [bend left] node[below] {$\mu$}      (1)
        edge [bend left] node[above] {$\lambda$}  (3)

    (3) edge [bend left] node[below] {$\mu$}      (2);

\end{tikzpicture}
}
	\caption{The Markovian chain describing the system}
\end{figure}

Considering local balance equations we get

\begin{align*}
  \lambda \pi_0 &= \mu \pi_1 \Leftrightarrow \pi_1 = \dfrac{\lambda}{\mu}\pi_0\\
  \lambda \pi_1 &= \mu \pi_2 \Leftrightarrow \pi_2 = \dfrac{\lambda}{\mu}\pi_1 = \Big(\dfrac{\lambda}{\mu}\Big)^2\pi_0\\
  \lambda \pi_2 &= \mu \pi_3 \Leftrightarrow \pi_3 = \Big(\dfrac{\lambda}{\mu}\Big)^3\pi_0\\
  \cdots
\end{align*}

In other words, in general $$\pi_i = \Big(\dfrac{\lambda}{\mu}\Big)^i\pi_0$$

In order to identify all $\pi_i$ we resort to the general law of
$$\sum_{i=0}^{\infty} \pi_i = 1$$

from where we calculate that
$$\pi_0 = 1 - \dfrac{\lambda}{\mu}$$

and

$$\pi_i = \Big(\dfrac{\lambda}{\mu}\Big)^i(1 - \dfrac{\lambda}{\mu})$$

The values for the intensities $\lambda$ and $\mu$ can be derived from

\begin{align*}
  \lambda &= \dfrac{1}{2 \cdot 10^{-3}} = 500 \\
  \mu     &= \dfrac{1}{10^{-3}} = 1000
\end{align*}

Hence the probability that the system is empty is given by

$$\pi_0 = 1 - \dfrac{\lambda}{\mu} = 0.5$$

and, in general, the probability that there are $i$ packets in the system
(in the queue and in the server) is

$$\pi_i = \Big(\dfrac{\lambda}{\mu}\Big)^i(1 - \dfrac{\lambda}{\mu}) = 0.5^{i+1}$$

The average number of packets waiting for transmission $N$ is the average number
of packets in the queue, hence

$$N = \sum_{i=2}^{\infty} i \cdot \pi_i = \sum_{i=2}^{\infty} i \cdot \Big(\dfrac{\lambda}{\mu}\Big)^i \cdot \pi_0 =
0.5 \sum_{i=2}^{\infty} i \cdot \Big(\dfrac{1}{2}\Big)^i = 0.5 \cdot 1.5 = 0.75$$

Where the starting condition is $i=2$ because waiting for transmission means that
there is at least one packet in the queue, and for that to happen there is
exactly one packet being transmitted in the server.
